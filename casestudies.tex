\documentclass[12pt,a4paper]{iopart}

% Document information
\newcommand{\DRN}{GridPP-ENG-001-CaseStudies}
\newcommand{\thisversion}{1.0}

% Author information
\input{common/authors/twhyntie.tex}

% Packages and settings for GridPP document formatting
% Packages and settings for GridPP document formatting.

% Graphics
\usepackage[xetex]{graphicx}
\usepackage{subcaption}
\usepackage{enumerate}

% Tables
\usepackage{array}
\usepackage{dcolumn}
\newcolumntype{.}{D{.}{.}{2}}
\newcolumntype{C}[1]{>{\centering\let\newline\\\arraybackslash\hspace{0pt}}m{#1}}
% For the width of the "Services" column.
\newcommand*{\servw}{1.0cm}

% Layout
\usepackage{layouts}
\usepackage{lscape}
\usepackage{parskip}

% Colours
\usepackage{xcolor}
\definecolor{black}{RGB}{0,0,0}
\definecolor{gridppdarkblue}{RGB}{11,45,131}
\definecolor{gridpplightblue}{RGB}{133,150,193}
\definecolor{cernblue}{RGB}{0,85,160}

% Mathematics
\usepackage{latexsym}

% Fonts
\usepackage{fontspec,xunicode}
\defaultfontfeatures{Mapping=tex-text,Scale=MatchLowercase}
\setmainfont{Latin Modern Sans}

% Headers and footers
%
% Using the fancyhdr package for the, um, fancy headings...
\usepackage{fancyhdr}
\setlength{\headheight}{16pt}
\setlength{\voffset}{-0.3in}
\pagestyle{fancy}
%
% Redefine the IOP article header settings for the first page.
\makeatletter
\def\ps@myheadings{%
%    \let\@oddfoot\@empty\let\@evenfoot\@empty
    \let\@oddhead\@empty\let\@evenhead\@empty
    \let\@mkboth\@gobbletwo
    \let\sectionmark\@gobble
    \let\subsectionmark\@gobble}
\makeatother
\lhead{\sffamily\rightmark}
\rhead{\sffamily\leftmark}
\lfoot{\sffamily{\DRN-v\thisversion}}
\cfoot{}
\rfoot{\sffamily{\thepage}}

% Exercises/questions
\usepackage[listings]{tcolorbox}
\tcbset{before={\par\medskip\pagebreak[0]\noindent},after={\par\medskip}}%

% Coding
\usepackage{listings}
% Listing options
\definecolor{listbacking}{rgb}{0.98,0.98,0.98}
\definecolor{commentgrey}{rgb}{0.60,0.60,0.60}
\definecolor{keywordblue}{rgb}{0.00,0.00,0.50}
\definecolor{stringgreen}{rgb}{0.00,0.50,0.00}
\definecolor{deepred}{rgb}{0.60,0.00,0.00}
\lstset{
  basicstyle=\footnotesize\ttfamily,
  gobble=4,
  float=hbtp,
  emph={__init__},
  emphstyle=\color{deepred},
  showspaces=false,
  showstringspaces=false,
  backgroundcolor=\color{listbacking},
  commentstyle=\color{commentgrey},
  stringstyle=\color{stringgreen},
  keywordstyle=\color{keywordblue},
  numbers=left,
  numbersep=12pt,
  %framexleftmargin=20pt,
  %framexrightmargin=20pt,
  stepnumber=1,
  showlines=true
}

% Acronyms
\usepackage{acronym}

\usepackage{perpage}
\MakePerPage{footnote}

% Flow charts and diagrams
\usepackage{tikz}
\usetikzlibrary{shapes,shapes.misc,arrows,backgrounds,positioning,fit,calc,shadows}
\def\checkmark{\tikz\fill[scale=0.4](0,.35) -- (.25,0) -- (1,.7) -- (.25,.15) -- cycle;}

% Background image.
\usepackage{eso-pic}
\newcommand\AtPageLowerRight[1]{\AtPageLowerLeft{%
   \makebox[\paperwidth][r]{#1}}}
\usepackage{ifthen}

% Links
\usepackage[xetex,pdfpagelabels]{hyperref}%
\hypersetup{%
pdfauthor={\theauthorfull},%
colorlinks=true,%
urlcolor=gridppdarkblue,%
citecolor=gridppdarkblue,%
%linkcolor=cernblue%
linkcolor=black%
}


% CERN@school macros and definitions
\input{common/tools/defs.tex}

%\input{common/tools/latexlinks.tex}

\renewcommand{\listfigurename}{}

\begin{document}

%
% Title
%
% The GridPP logo
\input{common/tools/logoheader.tex}

\title{%
The GridPP New User Engagement Programme:\\
Selected Case Studies
}

% 
% Author information
\author{\theauthorinit$^{1}$}
%
\address{$^1$\theauthoraddressA}
\ead{\mailto{\theauthoremail}}

%-----------------------------------------------------------------------------
% Abstract
\begin{abstract}
A number of selected case studies from the GridPP Collaboration's
New User Engagement Programme are presented,
featuring new user communities from fields including
medical physics, computational biology, space and astrophysics.
The services used and developed by GridPP
to achieve this engagement are also briefly
described.
\end{abstract}
%-----------------------------------------------------------------------------
%

% Add a table of contents.
\setcounter{tocdepth}{1}
\tableofcontents

% Add the license information (CC-BY-4).
\input{common/tools/licenseCCBY4}

\newpage

%%%%%%%%%%%%%%%%%%%%%%%%%%%%%%%%%%%%%%%%%%%%%%%%%%%%%%%%%%%%%%%%%%%%%%%%%%%%%%
\section{Introduction}
\label{sec:intro}
%%%%%%%%%%%%%%%%%%%%%%%%%%%%%%%%%%%%%%%%%%%%%%%%%%%%%%%%%%%%%%%%%%%%%%%%%%%%%%
The GridPP Collaboration~\cite{gridpp2006,gridpp2009}
is a community of particle physicists and computer
scientists based in the United Kingdom and at CERN.
Drawing on expertise from nineteen UK institutions,
its vision is to create, manage and oversee the evolution of the computing
infrastructure needed to maintain the UK's position as a world leader in
particle physics.
It does this by using and actively contributing to the development of
open source software, applications and middleware needed to power
large-scale distributed computing for particle physics and beyond.

GridPP was originally proposed to provide the UK's contribution to
computing required for the Large Hadron Collider~\cite{LHC2008} (LHC) at CERN.
%
However, GridPP's continually developing infrastructure has since
evolved to serve a wide range of user communities in many fields of
scientific and engineering endeavour.
Indeed, a requirement of GridPP's funding agreement is to make up
to 10\% of its resources available to non-LHC communities,
which as of 2012 was roughly 30,000 logical CPUs
(equating to 292,000 HEPSPEC06\footnote{%
See \href{https://w3.hepix.org/benchmarks/doku.php}{https://w3.hepix.org/benchmarks/doku.php}})
and 29 PB of storage.

%
With that in mind,
since 2013 the GridPP New User Engagement Programme has
encouraged the development and use of a number of tools and services
to enhance the potential for engagement with the Grid.
These include:

\begin{itemize}
%
\item \term{The GridPP CernVM}: The GridPP CernVM provides an out-of-the-box
Virtual Machine (VM) instance that is ready to be used as a gateway to the Grid.
Based on the Scientific Linux 6 operating system,
and using technology developed by CERN's
\href{http://cernvm.cern.ch}{CernVM group}~\cite{CernVM2015},
the GridPP CernVM is a contextualised VM image that has instant access to
the software and tools required to access GridPP's distributed computing
resources.
%
\item \term{The CernVM File System} (CernVM-FS or CVMFS): allows software to
be deployed, managed, and used wherever it is needed on the
Grid~\cite{CVMFS2015}.
%
\item \term{Ganga}: The Ganga software suite\footnote{%
See \href{http://ganga.web.cern.ch}{http://ganga.web.cern.ch}}
provides a Python-based interface to the Grid.
It was developed to provide a simpler interface to enable physicists to submit
their jobs across the Grid and to handle the complete life-cycle of each job.
Ganga can be configured to submit and manage jobs on a local machine,
a local batch system, or distributed GridPP resources via GridPP DIRAC.
%
\item \term{GridPP DIRAC}: The GridPP DIRAC service is a
DIRAC~\cite{DIRAC2010}
instance hosted at Imperial College London that has been developed to cater
for the multiple VOs that represent the non-LHC users of
GridPP resources~\cite{GRIDPPDIRAC2015a,GRIDPPDIRAC2015b}.
%
In combination with the Ganga software suite,
users have at their disposal a
powerful set of tools for managing complex distributed computing workflows.
\end{itemize}

A number of selected case studies, demonstrating how new user
communities have been engaged using this toolkit,
are
listed in Table~\ref{tab:casestudies} and
presented in the following sections of this document.


\begin{landscape}
{\color{white} There must be a better way of doing this.}
\\[3cm]

%______________________________________________________________________________
\begin{table}[h]
\caption{\label{tab:casestudies}A list of the new user community case studies
included in this document.}
\lineup
\begin{indented}
\item[]\begin{tabular}{@{}llC{\servw}C{\servw}C{\servw}C{\servw}C{\servw}C{\servw}}
\br
\centre{1}{$\quad$ User community $\quad$} &
\centre{1}{$\quad$ Sector         $\quad$} &
\centre{1}{$\quad$ Compute        $\quad$} &
\centre{1}{$\quad$ Storage        $\quad$} &
\centre{1}{$\quad$ CernVM         $\quad$} &
\centre{1}{$\quad$ CVMFS          $\quad$} &
\centre{1}{$\quad$ DIRAC          $\quad$} &
\centre{1}{$\quad$ Ganga          $\quad$} \\
\mr
LUCID (p\pageref{sec:lucid}) & Space & \checkmark & \checkmark & \checkmark & \checkmark & \checkmark & \\
HTC for Evolutionary Biology (p\pageref{sec:dolphins}) & Biology & \checkmark & \checkmark & & & & \\
Galactic Dynamics (UCLan) (p\pageref{sec:galdyn}) & Astrophysics & \checkmark & \checkmark & \checkmark & \checkmark & \checkmark & \\
PRaVDA - HTC for hadron therapy (p\pageref{sec:pravda}) & Medicine & \checkmark & & & \checkmark & \checkmark & \checkmark \\
The Large Synoptic Survey Telescope (p\pageref{sec:lsst}) & Astrophysics & \checkmark & \checkmark & & & \checkmark & \checkmark \\
Monopole hunting at the LHC (p\pageref{sec:moedal}) & Physics & \checkmark & \checkmark & \checkmark & \checkmark & \checkmark & \checkmark \\
\br
\end{tabular}
\end{indented}
\end{table}
%______________________________________________________________________________

\end{landscape}

\clearpage

%
% The case studies to include (renamed from autolist.tex).
%
\input{casestudies/autolucid}
\newpage
\input{casestudies/autodolphins}
\newpage
\input{casestudies/autogaldyn}
\newpage
\input{casestudies/autopravda}
\newpage
\input{casestudies/autolsst}
\newpage
\input{casestudies/automoedal}
\newpage


%
%%%%%%%%%%%%%%%%%%%%%%%%%%%%%%%%%%%%%%%%%%%%%%%%%%%%%%%%%%%%%%%%%%%%%%%%%%%%%%%
% Bibliography
%%%%%%%%%%%%%%%%%%%%%%%%%%%%%%%%%%%%%%%%%%%%%%%%%%%%%%%%%%%%%%%%%%%%%%%%%%%%%%%
%
\section*{References}
\bibliographystyle{unsrt.bst}
\bibliography{GridPP}
%
%------------------------------------------------------------------------------

\newpage

%%%%%%%%%%%%%%%%%%%%%%%%%%%%%%%%%%%%%%%%%%%%%%%%%%%%%%%%%%%%%%%%%%%%%%%%%%%%%%%
\section*{Acknowledgements}
\label{sec:ack}
%%%%%%%%%%%%%%%%%%%%%%%%%%%%%%%%%%%%%%%%%%%%%%%%%%%%%%%%%%%%%%%%%%%%%%%%%%%%%%%
The author wishes to thank the following people and organisations for
their support with the work described here:
%
the European Grid Initiative (EGI)
and
the Worldwide LHC Computing Grid (wLCG);
the CernVM group (CERN) for developing the CernVM
and CernVM-FS services that have revolutionised the way
users work with software on the Grid;
%
Catalin Condurache for supporting GridPP's CernVM-FS
capabilities at the Rutherford Appleton Laboratory (RAL);
%
Daniela Bauer, Simon Fayer (Imperial College London),
and Janusz Martyniak (now at RAL)
for their tremendous work developing and supporting GridPP DIRAC;
%
the Ganga development team\footnote{%
See \href{https://github.com/ganga-devs/}{https://github.com/ganga-devs/}}
for their work on and support for the Ganga user interface;
Sam Skipsey and Jens Jensen from the
GridPP Storage group for VO storage-related support;
%
Christopher J. Walker (QMUL), Alessandra Forti (Uni. Manchester)
and Andrew Lahiff (RAL) for their support
with setting up various new user communities at various points in the programme;
Andrew McNab (Uni. Manchester) with technical support with
the GridPP website and online user guide\footnote{%
See \href{https://www.gridpp.ac.uk/userguide/}{https://www.gridpp.ac.uk/userguide/}};
%
Dan Taylor, Alex Owen, Cozmin Timis, and Terry Froy (QMUL) for
support with development work on the Queen Mary University of London cluster;
%
Jeremy Coles (Uni. Cambridge) for leading GridPP Operations and
coordinating new user community activities;
%
Peter Gronbech (Uni. Oxford) for support with Project Management;
%
Suzanne Scott and Louisa Campbell (Uni. Glasgow) for administrative
support for Collaboration activities (particularly
the GridPP Collaboration meetings where the programme was
discussed and developed, as well as welcoming contributions from
newly engaged users);
%
David Britton (Uni. Glasgow) for leading the GridPP Collaboration
and supporting the vision of the New User Engagement Programme
as it developed, and, finally;
%
Steve Lloyd (QMUL) for chairing the GridPP Collaboration Board,
providing numerous testing and support mechanisms for Grid
activities, and many, many, many fruitful discussions.

This work was supported by the UK Science and Technology Facilities Council
(STFC) via the GridPP Collaboration~\cite{gridpp2006,gridpp2009}
and grant ST/N00101X/1 as part of work with the CERN@school research
programme.

\newpage

%%%%%%%%%%%%%%%%%%%%%%%%%%%%%%%%%%%%%%%%%%%%%%%%%%%%%%%%%%%%%%%%%%%%%%%%%%%%%%%
\section*{List of Figures}
%%%%%%%%%%%%%%%%%%%%%%%%%%%%%%%%%%%%%%%%%%%%%%%%%%%%%%%%%%%%%%%%%%%%%%%%%%%%%%%
Note that the images listed here are not subject to the
Creative Commons 4.0 lilcense under which this document (``the work'') is
issued. For information about licensing and re-use,
please contact the image owners as described in the figure captions.
\listoffigures

\newpage

%%%%%%%%%%%%%%%%%%%%%%%%%%%%%%%%%%%%%%%%%%%%%%%%%%%%%%%%%%%%%%%%%%%%%%%%%%%%%%%
\section*{Version History}
%%%%%%%%%%%%%%%%%%%%%%%%%%%%%%%%%%%%%%%%%%%%%%%%%%%%%%%%%%%%%%%%%%%%%%%%%%%%%%%
%______________________________________________________________________________
\begin{table}[h]
\caption{\label{tab:version}Version history.}
\lineup
\begin{indented}
\item[]\begin{tabular}{@{}cllc}
\br
\centre{1}{$\quad$Version    $\quad$} & 
\centre{1}{$\quad$Description$\quad$} &
\centre{1}{$\quad$DOI        $\quad$} &
\centre{1}{$\quad$Author     $\quad$} \\
\mr
1.0 & Initial version. & \href{http://dx.doi.org/10.5281/zenodo.220995}{10.5281/zenodo.220995} & TW \\
\br
\end{tabular}
\end{indented}
\end{table}
%______________________________________________________________________________

\end{document}
